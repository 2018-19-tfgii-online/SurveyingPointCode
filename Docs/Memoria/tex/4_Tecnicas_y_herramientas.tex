\capitulo{4}{Técnicas y herramientas}
\section{Metodologías}

Como metodología para el desarrollo del proyecto se ha utilizado \textbf {Scrum}.
Scrum es un marco de trabajo, es el método ágil de desarrollo de Software más utilizado del mundo.
Entre sus características principales están:
Utilizar una estrategia de desarrollo incremental, en lugar de la planificación y ejecución completa del producto. 
La calidad del resultado obtenido depende más del conocimiento tácito de las personas que de la calidad de los procesos usados. 
Este método permite el solapamiento entre diferentes fases del proyecto.
Algunos de sus componentes principales son: 
Sprint: Parte básica en este tipo de desarrollos, donde se desarrolla un incremento del producto que pueda ser utilizado.
Pila del producto: donde están los requisitos de usuario, está información no es rígida, puede variar según va evolucionando el producto.
Pila del sprint: lista con las tareas a realizar durante un sprint.
Incremento: Parte del desarrollo obtenida al final de cada sprint.

Durante todo el proyecto se han ido realizando los diferentes sprints, en los cuales se han planificado las tareas siguientes y revisado, si se han ido cumpliendo los objetivos marcados en el sprint anterior.


\section{Patrones de diseño}

Flask utiliza como patrón de diseño, \textbf {MTV}.
El patrón MTV es similar al conocido MVC, pero con algunas diferencias:

En el patrón MVC:
\begin{itemize}
\item\textbf{Modelo:} Es la parte que manipula la información.
\item\textbf{Vista:} Decide como se mostrará la información.
\item\textbf{Controlador:} Comunica el modelo con la vista.
\end{itemize}
Podemos ver a continuación sus equivalencias, para este caso, usando el framework Flask:


\begin{flushleft}
\textbf{\emph{MVC vs	MTV\\}}
Modelo=Modelo\\
Vista=Vista y Template\\
Controlador=Flask\\
\end{flushleft}



Aquí, el framework Flask, es el que toma el papel del controlador.
Resumiendo, en el patrón MTV, tenemos:
\begin{itemize}
\item\textbf{Modelo:} Es la parte que manipula la información, se encuentra en forma de clases de Python.
\item\textbf{Vista:} Decide qué información se muestra y en que template.
\item\textbf{Template:} Coge toda la información, la organiza y ve la manera en que esta se va a mostrar. Básicamente una página HTML con algunas etiquetas extras propias de Flask
\end{itemize}

\imagen{MTV}{Esquema de un patrón MTV}

En la imagen anterior podemos ver como son las relaciones en el patrón MTV:
\begin{enumerate}
\item El Navegador manda una solicitud 
\item La vista interactúa con el modelo para obtener datos. 
\item La vista llama a la plantilla. 
\item La plantilla renderiza la respuesta a la solicitud del navegador 
\end{enumerate}

\section{Control de Versiones}

Existen varias herramientas en el mercado para el control de versiones:
\begin{itemize}
\item Git
\item Subversion
\item SourceSafe
\end{itemize}

Nos hemos decantado por \textbf {Git}, ya que es la que se ha venido usando durante la realización del grado y es una de las mas extendidas en el mercado.

Git es un sistema de control de versiones distribuido, libre y de código abierto. Git se distribuye bajo la licencia de software libre GNU LGPL v2.1. 

\section{Hosting del repositorio}

Al igual que en el apartado anterior, aunque existen diversas posibilidades:
\begin{itemize}
\item GitLab
\item GitHub
\item Bitbucket
\end{itemize}

Como en el apartado anterior, por familiaridad con la plataforma durante el grado y también por ser una de las más usadas, de decidió usar \textbf {GitHub}.

GitHub permite tener cuentas gratuitas y de pago. Además, un tipo de cuenta para estudiantes, como alumno se puede solicitar acogerte al programa «GitHub education for students», que te permite tener repositorios privadas y da acceso a herramientas adicionales.

Además, GitHub ofrece otras funcionalidades y servicios como, por ejemplo; revisión de código, documentación, gestión de tareas, wikis e integraciones con otros servicios como por ejemplo CodeBeat.

\section{Gestión del proyecto}

Dentro de las muchas herramientas existentes como por ejemplo;
\begin{itemize}
\item ZenHub
\item Trello
\item VersionOne
\end{itemize}

Se ha optado por usar  \textbf {ZenHub} ya que se encuentra integrado en el propio GitHub, funciona como una aplicación nativa en su interfaz y podemos controlar nuestros proyectos usando paneles de trabajo bastante intuitivos, así como conectar con varios repositorios en el panel de tareas y ver todos los temas abiertos que requieren de la atención.

Se trata de una plataforma de gestión de proyectos de software y entre las principales características de ZenHub están la de convertir los issues de GitHub en dinámicos tablones Kanban.

\section{Entorno de desarrollo integrado (IDE)}


\subsubsection{Python}

Como en los apartados anteriores se han tenido en cuenta diferentes herramientas para este trabajo:
\begin{itemize}
\item NotePad ++
\item Sublime Text
\item Microsoft Visual Studio Code
\item PyCharm
\end{itemize}

Tanto Microsoft Visual Studio Code como PyCharm son perfectamente válidas y las más completas para realizar este tipo de proyectos, abarcan casi todos los lenguajes de programación, contienen bastantes plugins y facilitan el trabajo a la hora de realizar los test.

Al final nos hemos decantado por \textbf{PyCharm}. Al haberse realizado el proyecto con Python, este entorno ha sido diseñado para este lenguaje, por lo que ofrece algunas ventajas más, respecto a Microsoft Visual Studio Code. PyCharm es soportado por Windows, Mac OS y Linux, y su versión gratuita es bastante completa. Existen también licencias individuales para estudiantes.



\subsubsection{\LaTeX}
La documentación del proyecto ha desarrollado en \LaTeX. Para este propósito se han tenido en cuenta las siguientes herramientas:
\begin{itemize}
\item MikTex, para SO Windows
\item texlive, para SO Linux
\item MacTex, para SO Mac
\item Compatible con las 3 plataformas, Texmaker
\item Editor en línea, overleaf
\end{itemize}

Se ha optado por \textbf{Texmaker}, como recomendación del tutor del proyecto.
Integra la mayoría de herramientas necesarias para la escritura de documentos en \LaTeX, es gratuito y tiene licencia GNU GPL v2.

\section{Servicios de integración continua}


\subsubsection{Calidad del código}
Las diferentes herramientas consideradas para este propósito han sido: 
\begin{itemize}
\item SonarQube
\item CodeBeat
\end{itemize}

Se ha optado por \textbf{CodeBeat}, ya que se integra perfectamente con GitHub. Es una herramienta gratuita, que soporta gran cantidad de lenguajes y permite personalizar totalmente, las métricas que usa para el análisis de la calidad del código.

\section{Herramienta de análisis sintáctico}

Como herramienta de análisis sintáctico, para poder dar validez a los archivos de entrada, se pensó en estas 3 posibilidades:
\begin{itemize}
\item Ply
\item Antlr
\item Flex Bison
\end{itemize}

La opción elegida ha sido \textbf{Ply}, ya que está implementada completamente en Python y encaja perfectamente con la filosofía de realizar la mayor parte del proyecto con este lenguaje.

Aunque Ply también es una librería y debería estar en el apartado de librerías, le hemos querido dar un apartado especial como herramienta de análisis sintáctico.

Ply es una biblioteca de Python. Es una implementación de lex y yacc de herramientas de análisis para Python, que proporciona la mayor parte de las características estándar de lex / yacc, incluyendo; el apoyo a las producciones vacías, las reglas de prioridad, la recuperación de errores y soporte para gramáticas ambiguas. Es muy sencillo de usar y también muy efectivo a la hora de realizar la comprobación de los errores. Ply utiliza el análisis LR, el cual puede incorporar grandes gramáticas fácilmente.

Ply, básicamente está compuesto de 2 módulos:
\begin{itemize}
\item ply.lex – Donde se trata la parte del análisis léxico
\item ply.yacc -  Este módulo es para crear un parser.
\end{itemize}

\section{Librerías}

A continuación, vemos el resto de las librerías necesarias en el desarrollo de este proyecto:


\subsubsection{ezdxf}

ezdxf es la librería más importante del proyecto. En un principio, se tuvieron en cuenta otras opciones como; dxfgrabber, SDXF y  dxfwrite, resultando la opción más interesante ezdxf, ya que era la única que cubría todas las necesidades que aquí se planteaban.

\textbf{ezdxf} es una interfaz de Python para el formato DXF, desarrollado por Autodesk, y que permite a los desarrolladores leer y modificar dibujos DXF existentes o crear nuevos dibujos DXF.

Con sus métodos y atributos nos permite crear, modificar, eliminar o consultar las propiedades de todos los elementos que puede contener un archivo de CAD.
ezdxf es independiente del sistema operativo y se ejecuta en todas las plataformas que proporcionan un intérprete de Python adecuado (> = 3.6). Esta bajo la licencia MIT-License.

Una parte a destacar, es que nos permite generar el DXF en distintas versiones de un archivo CAD, lo cual es muy interesante. 

\imagen{Cad_versions}{Versiones de CAD}


\subsubsection{Bootstrap}

\textbf{Bootstrap} es un framework para desarrollo front-end. Permite crear de forma sencilla webs de diseño adaptable, es decir, que se ajusten a cualquier dispositivo y tamaño de pantalla y siempre se vean igual de bien.
Es una herramienta Open Source


\subsubsection{Colorpicker Bootstrap}

\textbf{Colorpicker Bootstrap},  es un plugin de selector de color modular para Bootstrap 4. Esta bajo la licencia MIT-License.


\subsubsection{TinyColor}

Es un micro framework que permite la manipulación y conversión de color en JavaScript. Permite muchas formas de entrada, mientras que proporciona conversiones de color y otras funciones de utilidad de color. No tiene dependencias. Esta bajo la licencia MIT-License


\subsubsection{bs-custom-file-input}

\textbf{bs-custom-file-input} ,es un plugin para Bootstrap 4. Ayuda a crear una entrada de selección de archivos personalizada con el botón del navegador para cargar archivos. También es compatible con múltiples selecciones de archivos y arrastrar y soltar.
No tiene dependencias. Esta bajo la licencia MIT-License

\section{Desarrollo Web}

Para el desarrollo web se ha optado por utilizar \textbf{Flask}, siguiendo, como se ha comentado anteriormente con la idea de englobar lo máximo posible en Python.
Flask es un “micro” Framework escrito en Python y concebido para facilitar el desarrollo de aplicaciones Web bajo el patrón MTV.

Entre algunas de sus ventajas, están:
\begin{itemize}
\item Desarrollo de aplicaciones web de una forma ágil y rápida, tiene una buena curva de aprendizaje
\item No se necesita una infraestructura con un servidor web.
\item Compatible con wsgi.
\item Soporta de manera nativa el uso de cookies seguras.
\item Se pueden usar sesiones.
\item Flask es Open Source y tiene una licencia BSD.
\item Extensa documentación.
\end{itemize}

\section{Despliegue de la aplicación}

Para el despliegue de la aplicación se ha optado por utilizar \textbf{Docker}. Principalmente, porque es una herramienta de mucha actualidad y con un gran potencial, tanto en el desarrollo, como en el despliegue de aplicaciones.

La idea básica de su funcionamiento es la creación de una serie de contenedores muy ligeros y portables, así las aplicaciones, podrán ejecutarse en cualquier equipo, solamente con tener Docker instalado, con independencia del sistema operativo.

Entre algunas de sus ventajas, están:

\begin{itemize}
\item Es muy sencillo de usar.
\item Ahorra tiempo, no necesitamos instalar diferentes softwares.
\item Son muy ligeros, a diferencia de las máquinas virtuales ya que no necesitan un sistema operativo completo, cogen lo que necesitan de cada máquina anfitriona.
\item Portabilidad.
\item Los repositorios Docker, nos permite acceder a gran cantidad de imágenes y así poder de ejecutar prácticamente todas las aplicaciones.
\item Es un entorno seguro y no ofrece variaciones, da igual cual sea el equipo ni el ambiente. Facilita las pruebas, el desarrollo y las visualizaciones por parte del cliente.
\item Es Open Source.
\item Extensa documentación.
\end{itemize}
