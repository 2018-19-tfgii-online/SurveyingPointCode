\capitulo{1}{Introducción}
\section{Introducción}
Uno de los trabajos que se realiza habitualmente en el campo de la Topografía y Cartografía, es la representación en un plano de los datos obtenidos en un levantamiento topográfico. Pongamos como ejemplo que se nos pide como producto final, un plano basado en un levantamiento topográfico de una zona de una ciudad para realizar el estudio de unas futuras obras. El plano debe contener todos los elementos existentes que sean de interés para este estudio, como, por ejemplo: edificios, aceras, viales, redes de saneamiento, abastecimiento, eléctricas, mobiliario urbano, etc.

Este trabajo se realizará en dos fases; la primera, el trabajo de campo, donde se realiza el levantamiento topográfico, adquisición de datos, que consiste en la medición georreferenciada de todos los elementos que sean de interés y que deben aparecer en el plano. La segunda fase es la de delineación, en la que se <<dibuja>> el plano. Hay que aclarar que, en la fase del levantamiento topográfico, solo se miden puntos, y es en la fase de delineación donde tenemos que crear todos los elementos, como líneas, curvas, símbolos, etc.   

Un problema importante al que nos enfrentamos es la gran cantidad de tiempo y recursos que invertimos en gestionar los datos obtenidos en campo para obtener un plano, alargando en exceso el desarrollo y entrega de un proyecto.
Con este planteamiento surge la necesidad de crear una aplicación que consiga automatizar este trabajo el máximo posible y así obtener un ahorro de tiempo y de recursos, fundamental en nuestros proyectos.
Puede parecer a priori un trabajo sencillo y rápido de realizar, con una herramienta de CAD, uniendo puntos manualmente para crear líneas, u otros elementos, claro está, si tienes un levantamiento topográfico con 30 puntos y recuerdas como lo has hecho en campo, pero cuando tienes un levantamiento con 1000 puntos, e intervienen varias personas en él, se puede convertir en un trabajo muy complicado e insufrible de realizar.
Una herramienta así no existe mercado, algunos programas como AutoCAD disponen de algún módulo a mayores del programa, que permiten hacer algo similar, pero con un resultado muy pobre respecto a lo que aquí se plantea.

La principal ventaja que aporta el uso de esta aplicación es el importante ahorro de tiempo y recursos que obtenemos en la realización de un trabajo. Por ejemplo, un trabajo de campo con 1000 puntos, con una codificación compleja, es decir con múltiples y diferentes elementos, una persona con habilidad en manejo de aplicaciones CAD, puede invertir alrededor de 2 horas. Esta aplicación pretende que ese proceso sea automático e inmediato. No debemos olvidar que el uso de una codificación totalmente aleatoria por el usuario a la hora de realizar el trabajo de campo también va a permitir obtener un importante ahorro de recursos, en esa parte del trabajo.
Otras ventajas son:
\begin{itemize}
	\item Es una aplicación web, por lo que los usuarios no necesitan instalar nada en sus equipos y pueden acceder desde cualquier lugar.
	\item Facilidad de uso, no hacen falta conocimientos técnicos en el uso de herramientas CAD.
\end{itemize}
\section{Contenido del proyecto}

La estructura de la memoria es la siguiente:

\begin{itemize}
	\item \textbf {Introducción:} breve descripción del contenido del trabajo y de la estructura de la memoria.
	\item \textbf {Objetivos del proyecto:} objetivos que se persiguen con la realización del proyecto.
	\item \textbf {Conceptos teóricos:} breve explicación de los conceptos necesarios para el desarrollo de la solución propuesta.
	\item \textbf {Técnicas y herramientas:} presentación de las técnicas metodológicas y las herramientas de desarrollo que se han utilizado para llevar a cabo el proyecto.
	\item \textbf {Aspectos relevantes del desarrollo:} descripción de los aspectos más importantes ocurridos a lo largo del desarrollo del proyecto.
	\item \textbf {Trabajos relacionados:} pequeño resumen de los trabajos y proyectos ya realizados en el campo del proyecto en curso.
	\item \textbf {Conclusiones y líneas de trabajo futuras:} resumen acerca de los conocimientos adquiridos y posibles aspectos de mejora o expansión de la solución aportada.
\end{itemize}

Además, se han desarrollado los siguientes anexos:

\begin{itemize}
	\item \textbf {Plan de Proyecto Software:} explicación sobre la planificación temporal llevada a cabo, así como un estudio de viabilidad del proyecto.
	\item \textbf {Especificación de requisitos:} descripción de la fase de análisis de la aplicación.
	\item \textbf {Especificación de diseño:} explicación y descripción del diseño de la aplicación.
	\item \textbf {Documentación técnica de programación:} recoge los aspectos más relevantes relacionados con el código fuente (estructura, compilación, instalación, ejecución, pruebas, etc.).
	\item \textbf {Documentación de usuario:} manual de usuario que permita conocer el funcionamiento la aplicación.
\end{itemize}

Por último, hay que indicar que el proyecto se encuentra disponible en el siguiente enlace:\\
 \url{https://github.com/EduardoRisco/SurveyingPointCode}
 
 
 
 