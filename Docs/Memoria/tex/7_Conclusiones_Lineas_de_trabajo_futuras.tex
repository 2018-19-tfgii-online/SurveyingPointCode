\capitulo{7}{Conclusiones y Líneas de trabajo futuras}

En este apartado se exponen las conclusiones derivadas del trabajo, así como
las posibles líneas de trabajo futuras por las que se puede dar continuidad al
proyecto y mejorar el producto.

\section{Conclusiones}

Una vez finalizado el proyecto, se han obtenido las siguientes conclusiones:

\begin{itemize}
\item El objetivo principal del proyecto se ha cumplido. Se ha desarrollado una aplicación que mediante un sistema de codificación de puntos aquí definida, consigue \textbf{automatizar el proceso de delineación} de un levantamiento topográfico. Convirtiendo un archivo de campo a un archivo DXF. Permitiendo al usuario poder subir una configuración personalizada y un archivo con símbolos. La aplicación también permite generar el archivo DXF en varias versiones de CAD. 

\item Se ha conseguido también, que el tipo de codificación definida, permita \textbf{mejorar el rendimiento en campo} a la hora de realizar el levantamiento topográfico. Mejoramos el rendimiento, al no tener que medir las líneas de forma consecutiva, creando cuadrados con solo dos puntos o rectángulos con tres puntos,etc, todo esto debido a la interpretación de la codificación.

\item No menos importante es resaltar que el usuario puede generar un plano en DXF, sin necesidad de tener conocimientos en programas de CAD.

\item En el proyecto se han utilizado los conocimientos adquiridos cursando el grado, en algunos casos requiriendo un conocimiento más profundo de estos. En otras disciplinas o tecnologías, como el desarrollo desarrollo de aplicaciones Web y el uso de Docker, se ha partido prácticamente de cero.
El tiempo dedicado a la investigación, seguro resultará una buena base para el futuro.

\item Se realizado este trabajo como si fuera un proyecto real,
utilizando la  metodología ágil \textbf{Scrum} en la gestión, comprobando las ventajas de su uso. Este tipo de metodología, ayuda a tener una visión más realista en cuanto a los objetivos que se desean conseguir, la planificación que se hace en los \emph{sprints} definiendo tareas alcanzables, hace que no pierdas el rumbo y vayas consiguiendo los objetivos paso a paso, ganando confianza cada vez que consigues los objetivos. 
\end{itemize}

\section{Líneas de trabajo futuras}

Aunque este trabajo como Trabajo Fin de Grado acaba aquí, habiendo desarrollado la parte mas compleja y difícil de la aplicación, y definiendo sistema para codificar los puntos de campo, el proyecto seguirá avanzando. \emph{SurveyingPointCode} pretende que el producto final sea un plano completo, añadiendo elementos como:



\begin{itemize}
\item Incluir marco en el plano.(Ver Figura~\ref{fig:marco})
\item Incluir cajetín en el plano.(Ver Figura~\ref{fig:cajetin})
\item Incluir leyenda en el plano.(Ver Figura~\ref{fig:leyenda})
\item Permitir al usuario elegir el formato del plano, A0, A1 ,A2, etc.
\item Permitir al usuario elegir la escala del plano, 1:1000 , 1:5000 , etc.
\end{itemize}

\imagen{marco}{Detalle marco de un plano}
\imagen{cajetin}{Detalle cajetín de un plano}
\imagen{leyenda}{Detalle leyenda de un plano}